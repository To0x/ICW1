\documentclass[ngerman,openany]{scrreprt}
\usepackage[latin1]{inputenc}

% ------------------------------ PAKETE ------------------------------  
\usepackage{units}
\usepackage{tabularx} % Paket f�r Tabellen
\usepackage{verbatim}
\usepackage{geometry} % Paket f�r Seitenr�nder u.�.
\usepackage{graphicx} % Paket f�r Grafiken
\usepackage{titlesec}
\usepackage{color}
\usepackage{fancyhdr} % Paket f�r Kopfzeile


% ------------------------------ KOPFZEILE ------------------------------  

\pagestyle{fancy} %eigener Seitenstil
\fancyhf{} %alle Kopf- und Fu�zeilenfelder bereinigen
\fancyhead[L]{\nouppercase{\leftmark}} %Kopfzeile links
\fancyhead[C]{} %zentrierte Kopfzeile
\fancyhead[R]{\thepage} %Kopfzeile rechts
\renewcommand{\headrulewidth}{0.4pt} %obere Trennlinie

% ------------------------------ DECKBLATT ------------------------------  
\begin{document}

\begin{titlepage}
\begin{center}
\Large{\textbf{Independent Coursework}} \\[8ex]
\LARGE{\textbf{Trends in Container-Virtulisierung}}\\[3ex]
\large{\quad} \\ %Schriftgr��e f�rs Deckblatt definieren (wenn diese Zeile entfernt wird ist alles weitere in LARGE
\includegraphics[scale=1.5]{HTW_Logo_rgb.jpg} \qquad \qquad \includegraphics[scale=1.0]{logo_desy.jpg} \\[10ex]
\begin{tabular}{l l l} \\
bearbeitet von: & \quad Tom Schubert \\[2ex]
Studiengang: & \quad Angewandte Informatik (Master) \\[2ex]
Fachbereich: & \quad Wirtschaftswisenschaften II \\[2ex]
Matrikelnummer: & \quad 535279 \\[2ex]
zust�ndiger Prof.: & \quad Prof. Dr. Hermann He�ling \\[2ex]
zust�ndige Mitarbeiter (DESY): & \quad Patrick Furhmann \\[2ex]
& \quad Yves Kemp \\[2ex]
Datum (Version): & \quad \today \quad (Ver. 0.0)
\end{tabular}
\end{center}
\end{titlepage}


% ------------------------------ INHALTSVERZEICHNIS ------------------------------  
\newpage
\thispagestyle{empty}
\tableofcontents
\newpage

\pagestyle{fancy} % F�r alle folgenden Seiten diesen Stil definieren
% ------------------------------ LANGTEXT ------------------------------  
\chapter{Grundlagen}
In diesem Kapitel werden die Grundlagen rund um das Thema Container erkl�rt. Hierzu wird darauf eingegangen, weshalb Container verwendet werden und welche Vorteile sie gegen�ber anderen Verfahren haben. \\
Verschiedene Containerarten werden gezeigt und erkl�rt, in welchen Aspekten diese sich unterscheiden und / oder �hneln.


\section{Container}



\subsection{Anwendungscontainer (kurz APPC)}

\subsection{Systemcontainer (kurz SC)}

\section{Images}

\subsection{App Container Image}
%ACI defined in https://github.com/appc/spec/blob/master/SPEC.md#app-container-image

\subsection{App Container Pod}

\subsection{Signed Images}
Image archives SHOULD be signed using PGP, the format MUST be ascii-armored detached signature mode. \\
Image signatures MUST be named with the suffix .aci.asc
% https://github.com/appc/spec/blob/master/SPEC.md

\section{Standardisierung}
gibt es keine. Versuch https://github.com/appc/spec


\section{Virtual Machines}
% im Prinzip genau das https://www.youtube.com/watch?v=_KnmRdK69qM
Sind zu langsam (Performance-Probleme)
Virtuelle Maschinen enthalten immer das komplette OS (Overhead)

\subsection{Unterschiede bzw. Vergleichbarkeit}


\section{OS-Spezifikationen}
Die meisten Container beruhen auf LXC (Linux).
Es gibt auch schon Windows-Container (wie funktionieren diese?)


\chapter{Containersoftware}
https://github.com/appc/spec

\section{LXC}
Linux-Grundlage f�r Container-Technologie, alles andere beruht darauf.

\section{LXD (SC)}

\section{Docker (APPC)}
http://codefest.at/post/2014/11/25/Erste-Schritte-mit-Docker-Teil-1.aspx

\section{Rocket (APPC)}

\section{Jetpack (APPC)}

\section{Libappc (APPC)}

\section{Kurma (APPC)}
http://www.apcera.com/blog/apcera-open-sources-new-kurma-project/

\chapter{Orchestration}

\section{Boot2Docker}

\section{VMWare}

\section{OpenStack}

\chapter{Performance}

\chapter{Security}

\end{document}


